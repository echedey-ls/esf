\chapter{Ejercicios de Geometría y Radiación Solar}


\section{Ángulos Solares}

\begin{itemize}
\item Calcula el Azimut, Ángulo Cenital y Altura Solar, Duración del
  Dia para el día del año 120, 2 horas después del mediodía, latitud
  $37.2\degree\mathrm{N}$:
  \[
    2\cdot\left|\omega_{s}\right|=\SI{13.5}{\hour}\]
  \[
    \gamma_{s}=55.1\degree\]
  \[
    \psi_{s}=57.7\degree\]

\item Calcula el Azimut, Ángulo Cenital y Altura Solar, Duración del
  Dia para el día del año día del año 340, 2 horas después del
  amanecer, latitud $\SI{15}{\degree}\mathrm{S}$:

\[
  2\cdot\left|\omega_{s}\right|=\SI{12.85}{\hour}\]
\[
  \gamma_{s}=27.13\degree\]
\[
  \psi_{s}=-108.2\degree\]

\item Calcula la duración del día 261 del año en las siguientes
  latitudes:

  \begin{itemize}
  \item $\SI{10}{\degree}\mathrm{N}$: $\SI{12.02}{\hour}$
  \item $\SI{40}{\degree}\mathrm{N}$: $\SI{12.11}{\hour}$
  \item $\SI{70}{\degree}\mathrm{N}$: $\SI{12.37}{\hour}$
  \item $\SI{10}{\degree}\mathrm{S}$: $\SI{11.97}{\hour}$
  \item $\SI{40}{\degree}\mathrm{S}$: $\SI{11.88}{\hour}$
  \item $\SI{70}{\degree}\mathrm{S}$: $\SI{11.63}{\hour}$
  \end{itemize}

\item Calcula la altura solar en el mediodía del día 25 del año en las
  latitudes:

  \begin{itemize}
  \item $\SI{10}{\degree}\mathrm{N}$: $\gamma_{s}=60.74\degree$
  \item $\SI{40}{\degree}\mathrm{N}$: $\gamma_{s}=30.74\degree$
  \item $\SI{10}{\degree}\mathrm{S}$: $\gamma_{s}=80.74\degree$
  \item $\SI{40}{\degree}\mathrm{S}$: $\gamma_{s}=69.26\degree$
  \end{itemize}

\item Calcula la hora solar real correspondiente al día 23 de Abril de
  2010 a las 12 de la mañana, hora oficial de la ciudad de A Coruña,
  Galicia. Esta localidad está contenida en el meridiano de longitud
  $8.38\degree\mathrm{W}$ y su hora oficial está regida por el huso
  horario GMT+1.

  Usamos la ecuación:
  
  \[
    \omega=15\cdot(\mathrm{TO}-\mathrm{AO}-12)+\Delta\lambda+\frac{\mathrm{EoT}}{4}
  \]

  En España se aplica el horario de verano y este día está incluido en
  el período afectado, $\mathrm{AO}=1$.

  En cuanto a las longitudes:
  \begin{align*}
    \lambda_{L} &= -8.38\degree\\
    \lambda_{H} &= 15\degree\\
    \Delta\lambda &= -23.38\degree
  \end{align*}
  

  Por último, para este día la ecuación del tiempo da como resultado:

  \begin{align*}
    d_n &= 113\\
    M &=\frac{2\pi}{365.24}\cdot 113 = 1.9439\\
    \mathrm{EoT}&=\mathrm{EoT}=229.18\cdot\left(-0.0334\cdot\sin(1.9439) + 0.04184\cdot\sin\left(2\cdot
                  1.9439 + 3.5884\right)\right) = \SI{1.785}{\minute}
  \end{align*}
  
  Por tanto, $\omega=-37.94\degree$ (aproximadamente las 9 y media de
  la mañana). El Sol culminará ($\omega=0$) cuando sean las 14:31,
  hora oficial.

\end{itemize}



\section{Componentes de irradiación en el plano horizontal}


\begin{itemize}
\item Calcula la irradiación diaria extra-atmosférica en el plano
  horizontal del día 18 de septiembre ($d_n = 261$) en un lugar de
  latitud 40\degree N.

\begin{align*}
  \delta &= 23,45\degree\cdot\sin\left(\frac{2\pi\cdot\left(261+284\right)}{365}\right) = 1.0089\degree\\
  \\
  \cos \omega_s &= -\tan(\delta) \tan(\phi) = -0.01478\\
  \omega_s &= -90.85\degree = \SI{-1.586}{\radian}\\
  \\
  \epsilon_0 &= 1+0.033\cdot\cos(2\pi\cdot261/365) = 0.9928151\\
  \\
  B_{0d}(0) &= -\frac{24}{\pi}B_{0}\epsilon_{0}\cdot\left(\omega_{s}\sin\phi\sin\delta+\cos\delta\cos\phi\sin\omega_{s}\right) = \\
         &= \SI{8126.37}{\watthour\per\meter\squared}
\end{align*}

\item Calcula las medias mensuales de las componentes de la
  irradiación en un lugar de latitud 40\degree N, cuya media mensual
  de irradiación global en el plano horizontal en el mes de septiembre
  es de $G_{d,m}(0) = \SI{4150}{\watthour\per\meter\squared}$:

  La media mensual de la irradiación extra-atmosférica diaria en el
  mes de septiembre equivale a la irradiación extra-atmosférica en el
  día 261 (día promedio), calculado en el apartado anterior:

\[
  B_{0d,m}(0) = \SI{8126.37}{\kilo\watthour\per\meter\squared}
\]

Por tanto, el índice de claridad en este mes es:

\[
  K_{Tm}=\frac{4150}{8126.37}=0.5107
\]

Según la correlación de Page (medias mensuales):

\[
  F_{Dm} = 1 - 1.13 \cdot 0.5107 = 0.4229
\]

Por tanto, las componentes difusa y directa son:

\begin{align*}
  D_{d,m}(0) &= 0.4229 \cdot 4150 = \SI{1755.04}{\watthour\per\meter\squared}\\
  B_{d,m}(0) &= 4150 - 1755.04 = \SI{2394.97}{\watthour\per\meter\squared}  
\end{align*}


\item Calcula las componentes directa y difusa de la radiación solar
  del 17 de Septiembre (día 261) en un lugar con latitud
  $\phi=\ang{40}\mathrm{N}$ y con irradiación global diaria horizontal
  $G_d(0)=\SI{4510}{\watthour\per\meter\squared}$.

\begin{align*}
  B_{0d} &= \SI{8126.366}{\watthour\per\meter\squared}\\
  K_{Td} &= G_{d}/B_{0d} = \frac{2700}{8126.366} = 0.55498\\
  F_{Dd} &= 1.188 - 2.272 \cdot 0.55498 + 9.473 \cdot 0.55498^{2} -\\
         &- 21.856 \cdot 0.55498^{3} + 14.648 \cdot 0.55498^{4} = 0.4984\\
  D_d(0) &= F_{Dd} \cdot G_d(0) = 0.4984 \cdot 4510 = \SI{2247.78}{\watthour\per\meter\squared}\\
  B_d(0) &= G_d(0) - D_d(0) = \SI{2262.22}{\watthour\per\meter\squared}
\end{align*}


\item Calcula la irradiancia global y la irradiancia difusa en el
  plano horizontal 2 horas antes del mediodía del día 261 en un lugar
  con latitud $\phi=\ang{40}\mathrm{N}$ y con media mensual de
  irradiación global diaria horizontal
  $G_{d,m}(0)=\SI{4150}{\Wh\per\meter\squared}$.

  El valor de $\omega$ 2 horas antes del mediodía es
  $\omega = \SI{-2}{\hour} = -30\degree.$

\begin{align*}
  r_{D} &= \frac{\pi}{24}\cdot\frac{\cos(\omega)-\cos(\omega_{s})}{\omega_{s}\cdot\cos(\omega_{s})-\sin(\omega_{s})} = 0.113\\
  a &= 0.409-0.5016\cdot\sin(\omega_{s}+\frac{\pi}{3})\\
  b &= 0.6609+0.4767\cdot\sin(\omega_{s}+\frac{\pi}{3})\\
  r_{G} &= r_{D}\cdot\left(a + b\cdot\cos(\omega)\right) = 0.116
\end{align*}

\begin{align*}
  D(0) &= r_D \cdot D_{d}(0) = 0.112 \cdot 1755.04 = \SI{197.98}{\watt\per\meter\squared}\\
  G(0) &= r_G \cdot G_{d}(0) = 0.116 \cdot 2394.97 = \SI{277.54}{\watt\per\meter\squared}\\
  B(0) &= G(0) - D(0) = \SI{79.56}{\watt\per\meter\squared}
\end{align*}
\end{itemize}
\section{Ángulos y componentes de irradiación en Sistemas
  Fotovoltaicos}

\begin{itemize}
\item Calcula el ángulo de incidencia para el
  \begin{itemize}
  \item Día del Año: 120, 2 horas después del mediodía, Latitud:
    $\SI{37.2}{\degree}\mathrm{N}$;

    \begin{itemize}
    \item Un sistema estático orientado al Sur y con inclinación de
      $\SI{30}{\degree}$: $\theta_{s}=30.33\degree$
    \item Un sistema de seguimiento horizontal N-S:
      $\theta_{s}=17.98\degree$
    \item Un sistema de seguimiento acimutal con inclinación a
      $\SI{35}{\degree}$: $\theta_{s}=0\degree$
    \item Un sistema de seguimiento a doble eje: $\theta_{s}=0\degree$
    \end{itemize}
  \item Día del Año: 340, 2 horas después del amanecer, Latitud:
    $\SI{15}{\degree}\mathrm{S}$;

    \begin{itemize}
    \item Un sistema estático orientado al Sur y con inclinación de
      $\SI{30}{\degree}$: $\theta_{s}=75.1\degree$
    \item Un sistema de seguimiento horizontal N-S:
      $\theta_{s}=16.1\degree$
    \item Un sistema de seguimiento acimutal con inclinación a
      $\SI{35}{\degree}$: $\theta_{s}=27.87\degree$
    \item Un sistema de seguimiento a doble eje: $\theta_{s}=0\degree$
    \end{itemize}
  \end{itemize}


\item Calcula la irradiancia difusa, directa, de albedo y global, en
  un generador inclinado $\ang{30}$ y orientado al Sur, 2 horas antes
  del mediodía del día 261 en un lugar con latitud
  $\phi=\ang{40}\mathrm{N}$ y con media mensual de irradiación global
  diaria horizontal $G_{d,m}(0)=\SI{4150}{\Wh\per\meter\squared}$.

  En primer lugar obtenemos las variables de geometría solar:
  \begin{align*}
    \cos(\theta_{zs}) &= \cos(\delta) \cos(\omega) \cos(\phi) + \sin(\delta) \sin(\phi) = 0.6754\\
    B_{0}(0) &=B_{0}\cdot\epsilon_{0}\cdot\cos\theta_{zs} = \SI{916.69}{\watt\per\meter\squared}
  \end{align*}

  A continuación, obtenemos el coseno del ángulo de incidencia:

\[
  \cos(\theta_{s}) = \cos(\delta)\cos(\omega)\cos(\beta-|\phi|)-
  \mathrm{sign}(\phi)\cdot\sin(\delta)\sin(\beta-|\phi|) = 0.8568
\]

Además, aprovechamos los valores de las componentes de irradiancia en
el plano horizontal obtenidos en un apartado anterior:

\begin{align*}
  D(0) &= \SI{197.98}{\watt\per\meter\squared}\\
  G(0) &= \SI{277.54}{\watt\per\meter\squared}\\
  B(0) &= \SI{79.56}{\watt\per\meter\squared}
\end{align*}

Con este conjunto de valores podemos estimar las componentes de
irradiancia:

\[
  B(\beta,\alpha) =
  B(0)\cdot\frac{\max(0,\cos(\theta_{s}))}{\cos(\theta_{zs})} =
  \SI{100.93}{\watt\per\meter\squared}
\]

\begin{align*}
  k_{1} &= \frac{B(0)}{B_{0}(0)} = 0.087\\
  D^{I}(\beta,\alpha) &= D(0) \cdot (1-k_{1}) \cdot \frac{1 + \cos(\beta)}{2} = \SI{168.68}{\watt\per\meter\squared}\\
  D^{C}(\beta,\alpha) &= D(0) \cdot k_{1} \cdot \frac{\max(0,\cos(\theta_{s}))}{\cos(\theta_{zs})} = \SI{21.79}{\watt\per\meter\squared}\\
  D(\beta,\alpha) &= D^{I}(\beta,\alpha)+D^{C}(\beta,\alpha) = \SI{190.48}{\watt\per\meter\squared}
\end{align*}

\begin{align*}
  \rho &= 0.2\\
  R(\beta,\alpha) &= \rho\cdot G(0)\cdot\frac{1-\cos(\beta)}{2} = \SI{3.72}{\watt\per\meter\squared}
\end{align*}

\[
  G(\alpha,\beta) = B(\alpha, \beta) + D(\alpha, \beta) + G(\alpha,
  \beta) = \SI{294.13}{\watt\per\meter\squared}
\]
\end{itemize}
