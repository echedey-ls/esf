\usepackage[T1]{fontenc}
\usepackage[utf8]{inputenc}
\usepackage[a4paper]{geometry}
\geometry{verbose,tmargin=2.5cm,bmargin=2.5cm,lmargin=2.5cm,rmargin=2.5cm}
\pagestyle{Ruled}
\usepackage{array}
\usepackage{verbatim}
\usepackage{prettyref}
\usepackage{booktabs}
\usepackage{textcomp}
\usepackage{url}
\usepackage{amsmath}
\usepackage{chemarr}%flechas para reacciones químicas (SFER.tex)
\usepackage{graphicx}
\usepackage{amssymb}
\usepackage{nomencl}
\usepackage[usenames,dvipsnames]{xcolor}

% the following is useful when we have the old nomencl.sty package
% \providecommand{\printnomenclature}{\printglossary}
% \providecommand{\makenomenclature}{\makeglossary}
\makenomenclature

\usepackage[caption=false]{subfig}
%Configuración de los caption
%\PassOptionsToPackage{caption=false}{subfig}%Evita que el paquete subfig lo descabale todo
\captiontitlefont{\itshape}
\captionnamefont{\scshape}
%\captionstyle{\centering}
\hangcaption


\usepackage[spanish]{babel}
\addto\shorthandsspanish{\spanishdeactivate{~<>}}


\usepackage[emulate=units]{siunitx}
\newunit{\wattpeak}{Wp}
\newunit{\watthour}{Wh}
\newunit{\amperehour}{Ah}
\sisetup{per-mode=fraction, fraction-function=nice, output-decimal-marker=comma}
%\usepackage{lscape}
\usepackage{mathpazo}%Letra palatino con fuentes para matemáticas
\usepackage{flafter}%obliga a que los flotantes aparezcan después de su referencia
\usepackage{memhfixc}


\raggedbottom
\sloppybottom
\clubpenalty=10000
\widowpenalty=10000

%\raggedbottomsection
\feetbelowfloat


\usepackage[citestyle=alphabetic, bibstyle=alphabetic, maxbibnames=5,minbibnames=3,
backend=bibtex, doi=true, url=true]{biblatex}

\DefineBibliographyStrings{spanish}{%
  andothers        = {et\addabbrvspace al\adddot},
  andmore          = {et\addabbrvspace al\adddot},
  in               = {},
}

\addbibresource{../biblio.bib}

\let\cite\parencite

\renewcommand{\bibsection}{%
	\chapter*{\bibname}
	\bibmark
	\phantomsection
	\addcontentsline{toc}{chapter}{\bibname}
	\prebibhook}
% \renewcommand{\bbltechreport}{Informe T\'ecnico}

\usepackage{hyperref}


\hypersetup{
    bookmarks=true,         % show bookmarks bar?
    unicode=true,          % non-Latin characters in Acrobat’s bookmarks
    bookmarksnumbered=false,
    bookmarksopen=false,
    breaklinks=true,
    backref=true,
    pdftoolbar=true,        % show Acrobat’s toolbar?
    pdfmenubar=true,        % show Acrobat’s menu?
    pdffitwindow=false,     % window fit to page when opened
    pdfstartview={FitH},    % fits the width of the page to the window
    pdftitle={Energía Solar Fotovoltaica},    % title
    pdfauthor={Oscar Perpiñán Lamigueiro},     % author
    pdfsubject={Energia Solar Fotovoltaica},   % subject of the document
    pdfcreator={AucTeX/Emacs},   % creator of the document
    pdfproducer={LaTeX}, % producer of the document
    pdfkeywords={radiación solar, energía solar fotovoltaica, energías
    renovables}, % list of keywords
    pdfnewwindow=true,      % links in new window
    pdfborder={0 0 0},
    colorlinks=true,       % false: boxed links; true: colored links
    linkcolor=Brown,          % color of internal links
    citecolor=BrickRed,        % color of links to bibliography
    filecolor=black,      % color of file links
    urlcolor=Blue           % color of external links 
}
	

\DeclareSIUnit\kWh{kWh}
\DeclareSIUnit\Wh{Wh}
\DeclareSIUnit\Wp{Wp}
\DeclareSIUnit\kWp{kWp}
\DeclareSIUnit\amperehour{Ah}
\DeclareSIUnit\celula{celula}

%\spanishdecimal{.} %Para que no lo sustituya automáticamente por comas
\addto\captionsspanish{%
\def\tablename{Tabla}%
\def\listtablename{\'Indice de tablas}}

\renewcommand\nomname{Nomenclatura}
\def\nompreamble{\addcontentsline{toc}{chapter}{\nomname}\markboth{\nomname}{\nomname}}


%\@addtoreset{equation}{section}
%\renewcommand{\theequation}{\thesection.\arabic{equation}}
%\numberwithin{equation}{section}
%\@addtoreset{table}{section}
%\renewcommand{\thetable}{\thesection.\arabic{table}}
%\numberwithin{table}{section}
%\@addtoreset{figure}{section}
%\renewcommand{\thefigure}{\thesection.\arabic{figure}}
%\numberwithin{figure}{section}


%\declarebtxcommands{spanish}{%
 % \def\btxphdthesis#1{\protect\foreignlanguage{spanish}{Tesis Doctoral}}%
%}
%\setbibliographyfont{lastname}{\scshape}%Pone los autores en Small Caps



%Configuración de MEMOIR
%%Pone la fecha en SMALL CAPS y hacia la derecha
%%pagina 60 de memman.pdf
\pretitle{\vfill \begin{flushright} \bfseries \scshape \HUGE \color{BrickRed}}
\posttitle{\par\end{flushright}}

\preauthor{\begin{flushright} \large \scshape}
\postauthor{\par\end{flushright}}

%\date{}
\predate{\vfill \begin{flushright}\large\scshape}
\postdate{\par\end{flushright}\vfill}

\setsecnumdepth{subsection}


% \definecolor{ared}{rgb}{.647,.129,.149}
% \renewcommand{\colorchapnum}{\color{ared}}
% \renewcommand{\colorchaptitle}{\color{ared}}
% \chapterstyle{pedersen}
\chapterstyle{ger}

\setlength{\afterchapskip}{35pt}
\maxtocdepth{section}

%\setcounter{topnumber}{3}
%\setcounter{bottomnumber}{2}
%\setcounter{totalnumber}{4}
\renewcommand{\topfraction}{0.85}
\renewcommand{\bottomfraction}{0.5}
\renewcommand{\textfraction}{0.15}
\renewcommand{\floatpagefraction}{0.7}


%Centra las figuras en los flotantes y los enmarca
\makeatletter
\renewenvironment{figure}[1][]{%
     	\@float{figure}%
		%\begin{framed}    
		\precaption{\rule{\linewidth}{0.4pt}\par}%En las figuras el caption va debajo
		%\hrule\vspace{\onelineskip}
		\centering
		  }{%
		%\end{framed}
		%\postcaption{\rule{\linewidth}{0.4pt}}
		%\vspace{\onelineskip}\hrule
    	\end@float	
}
\makeatother

\makeatletter
\renewenvironment{table}[1][]{%
      	\@float{table}%
		%\begin{framed}    
		\postcaption{\rule{\linewidth}{0.4pt}\par}%En las tablas el caption va encima
		\centering
		  }{%
		%\end{framed}
    	\end@float	
}
\makeatother


\renewcommand{\textfloatsep}{10pt}%Espacio entre el flotante y el texto

%backgroung image
\usepackage{eso-pic} 

%http://stackoverflow.com/questions/240097/how-to-create-a-background-image-on-titlepage-with-latex
\newcommand\BackgroundPic{
  \put(350,-150){
    \parbox[b][0.5\paperheight]{0.5\paperwidth}{%
      \includegraphics[scale=0.5]{../figs/johnny_automatic_old_sun}%
}}}

\newcommand\BackgroundPicLight{
  \put(350,-150){
    \parbox[b][0.5\paperheight]{0.5\paperwidth}{%
 %     \vfill 
%\centering
      \includegraphics[scale=0.5]{../figs/johnny_automatic_old_sun_light}%
%\vfill
}}}

